\documentclass[a4paper,11pt]{article}
\usepackage{amsmath,amsthm,amsfonts,amssymb,bm} 
\usepackage{graphicx,psfrag} 
\usepackage{fancyhdr}
\usepackage{color} 
\usepackage{geometry}
\usepackage{multirow}
\usepackage{listings}
\usepackage{enumerate}
\usepackage{leftidx} 
\usepackage{mathrsfs} 
\usepackage{xeCJK}

\usepackage{listings}
\usepackage{color}

\definecolor{mygreen}{rgb}{0,0.6,0}
\definecolor{mygray}{rgb}{0.5,0.5,0.5}
\definecolor{mymauve}{rgb}{0.58,0,0.82}

\lstset{ %
  backgroundcolor=\color{white},   % choose the background color; you must add \usepackage{color} or \usepackage{xcolor}
  basicstyle=\footnotesize,        % the size of the fonts that are used for the code
  breakatwhitespace=false,         % sets if automatic breaks should only happen at whitespace
  breaklines=true,                 % sets automatic line breaking
  captionpos=b,                    % sets the caption-position to bottom
  commentstyle=\color{mygreen},    % comment style
  deletekeywords={...},            % if you want to delete keywords from the given language
  escapeinside={\%*}{*)},          % if you want to add LaTeX within your code
  extendedchars=true,              % lets you use non-ASCII characters; for 8-bits encodings only, does not work with UTF-8
  keepspaces=true,                 % keeps spaces in text, useful for keeping indentation of code (possibly needs columns=flexible)
  keywordstyle=\bfseries,       % keyword style
  language=Ruby,                 % the language of the code
  morekeywords={*,...},            % if you want to add more keywords to the set
  numbers=none,                    % where to put the line-numbers; possible values are (none, left, right)
  numbersep=5pt,                   % how far the line-numbers are from the code
  numberstyle=\tiny\color{mygray}, % the style that is used for the line-numbers
  rulecolor=\color{black},         % if not set, the frame-color may be changed on line-breaks within not-black text (e.g. comments (green here))
  showspaces=false,                % show spaces everywhere adding particular underscores; it overrides 'showstringspaces'
  showstringspaces=false,          % underline spaces within strings only
  showtabs=false,                  % show tabs within strings adding particular underscores
  stepnumber=2,                    % the step between two line-numbers. If it's 1, each line will be numbered
  stringstyle=\color{mymauve},     % string literal style
  tabsize=2,                       % sets default tabsize to 2 spaces
  title=\lstname                   % show the filename of files included with \lstinputlisting; also try caption instead of title
}

\setCJKmainfont[BoldFont=兰亭黑-简,ItalicFont=STKaiti]{STSong}
\geometry{left=3.17cm,right=3.17cm,top=2.54cm,bottom=2.54cm}

\begin{document}

\pagestyle{fancy}
\rfoot{\thepage}
\rhead{\bfseries 计算机网络}
\setlength{\parskip}{0.7ex plus0.2ex minus0.2ex}
\cfoot{\empty}
\lhead{\empty}


\title{FTP实验报告}
\author{李青林,5110307074}
\date{}
\maketitle

\headheight 3pt
\thispagestyle{fancy}
\section{实验概况}
本实验利用Ruby编程语言在Unix-like系统上完成了一个文件传输协议(FTP)的简单实现,利用Socket接口在应用层上通过TCP协议实现了FTP协议.
\section{实验内容}
\subsection{Server}
Server需首先开启一个TCPSocket监听某一端口(默认为21,可配置),默认的根目录为同目录下的root文件夹.
对于每一个请求,开启一个进程处理该请求.
该进程将开启一个死循环,若接收到的信息(命令)不为空,则将该命令交给一个CMDHandler类处理,否则跳出死循环.

CMDHandler类负责具体处理每条客户端发来的命令,维护用户当前目录等信息,目前支持如下命令
\begin{enumerate}
\item \textbf{CWD}
更改CMDHandler类记录的当前目录
\item \textbf{PWD}
根据CMDHandler类记录的当前目录,返回该信息给用户
\item \textbf{PORT}
根据PORT命令中的相关信息,连接Client端的相应端口
\item \textbf{RETR}
读取相应文件,利用PORT命令建立的数据连接将文件传送给Client端
\item \textbf{LIST}
读取当前目录下地文件列表,利用PORT命令建立的数据连接将文件列表发送给Client端
\item \textbf{QUIT}
返回Goodbye并关闭所有TCP连接
\item \textbf{USER}
目前仅支持你们登录,即对所有USER命令返回登录成功
\item \textbf{SYST}
返回当前系统为UNIX
\item \textbf{DELE}
调用相应API尝试删除文件,并将结果返回给用户
\item \textbf{MDTM}
调用相应API尝试获取文件修改时间,并将结果返回给用户
\item \textbf{MKD}
调用相应API尝试建立目录,并将结果返回给用户
\item \textbf{RMD}
调用相应API尝试删除目录,并将结果返回给用户
\item \textbf{SIZE}
调用相应API尝试获取文件大小,并将结果返回给用户
\item \textbf{STOR}
从PORT命令建立的数据连接中获取文件数据,并保存在本地
\item \textbf{RNFR}
记录要改名的文件,将该文件存在与否的信息返回给用户
\item \textbf{RNTO}
调用相应API尝试文件改名,将改名结果返回给用户
\end{enumerate}
\subsection{Client}
Client通过建立一个TCPSocket,连接相应Server的相应端口,并进入一个死循环,每次接受用户输入的命令,并将该命令转交给Handler类执行.

Handler类负责具体处理用户输入的命令,为了方便用户使用,我自己根据Unix-like系统的文件操作命令及系统FTP的相应命令设计了如下一套客户端命令
\begin{enumerate}
    \item \textbf{pwd, cd, user, mkdir, rmdir, rm}
        这几条命令的处理方式类似,都是发送相应的FTP命令及参数给Server,并显示执行结果.
        其对应的FTP命令分别为PWD,CWD,USER,MKD,RMD,DELE.
    \item \textbf{mv}
        对于mv命令,首先发送一条RNFR命令,若执行成功,则发送一条RNTO命令,并将执行结果返回给用户.
    \item \textbf{ls}
        首先发送一条PORT命令,与Server建立一个数据连接,利用该连接获取文件列表并返回给客户,最后关闭数据连接.
    \item \textbf{get}
        首先发送一条PORT命令,与Server建立一个数据连接,利用该连接获取文件内容并存储在本地,返回给客户执行结果,最后关闭数据连接.
    \item \textbf{put}
        首先发送一条PORT命令,与Server建立一个数据连接,读取文件内容,并利用该连接将文件内容发送给Server,将执行结果返回给用户,最后关闭数据连接.
    \item \textbf{bye}
        发送QUIT给Server,关闭所有连接并退出客户端.
\end{enumerate}
\section{实验总结}
通过这次实验,我理解了FTP协议的原理和协议细节,学习了利用Socket接口设计实现简单应用层协议,掌握了TCP网络应用程序的基本设计方法.

\end{document}
